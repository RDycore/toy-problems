\documentclass{article}

\usepackage{amsmath}
\usepackage{esint}
\usepackage[a4paper, total={6in, 8in}]{geometry}
\usepackage[subdued,defaultmathsizes]{mathastext}

\newcommand{\p}{{\partial}}

\def\EQ#1\EN{\begin{equation}#1\end{equation}}
\def\BA#1\EA{\begin{align}#1\end{align}}
\def\BS#1\ES{\begin{split}#1\end{split}}
\def\EQA#1\ENA{\begin{eqnarray}#1\end{eqnarray}}
\def\SEQ#1\SQN{\begin{subequations}#1\end{subequations}}

\begin{document}

\section{Governing equations}

The shallow water equations are given by
\EQ
\label{eqn:swe}
\frac{\p\mathbf{U}}{\p t} + \frac{\p \mathbf{E}}{\p x} + \frac{\p \mathbf{G}}{\p x} = \mathbf{S}
\EN

where 
\begin{align}
	\mathbf{U}
	=
	\begin{bmatrix}
	h \\[.5em]
	hu \\[.5em]
	hv
	\end{bmatrix}
\end{align}

\begin{align}
	\mathbf{E}
	=
	\begin{bmatrix}
	hu \\[.5em]
	hu^2 + \frac{1}{2}gh^2 \\[.5em]
	huv
	\end{bmatrix}
\end{align}

\begin{align}
	\mathbf{G}
	=
	\begin{bmatrix}
	hv \\[.5em]
	huv \\[.5em]
	hv^2 + \frac{1}{2}gh^2
	\end{bmatrix}
\end{align}

\begin{align}
	\mathbf{S}
	=
	\begin{bmatrix}
	Q_i \\[.5em]
	-gh\frac{\p z}{\p x} - C_D u \sqrt{u^2 + v^2}\\[.5em]
	-gh\frac{\p z}{\p y} - C_D v \sqrt{u^2 + v^2}
	\end{bmatrix}
\end{align}
%
$h$ is the flow depth,
$u$ is the vertically-averaged velocity in x-direction,
$v$ is the vertically-averaged velocity in x-direction,
$z$ is the bed elevation,
$C_D = g n^2/h^\frac{1}{3}$ is the drag coefficient, and
$n$ is the Manning's coefficient.

\section{Spatial discretization}

\EQA
\frac{\p}{\p t} \int_\Omega \mathbf{U} d\Omega + 
\int_\Omega \frac{\p \mathbf{E}}{\p x}  d\Omega + 
\int_\Omega \frac{\p \mathbf{G}}{\p x}  d\Omega +  &=&
\int_\Omega \mathbf{S} d\Omega \nonumber\\
%
\frac{\p}{\p t} \int_\Omega \mathbf{U} d\Omega + 
\oint_{d\Omega} \left( \mathbf{E}dy  - \mathbf{G} dx \right) ds &=&
\int_\Omega \mathbf{S} d\Omega \nonumber\\
%
\frac{\p}{\p t} \int_\Omega \mathbf{U} d\Omega + 
\int_{d\Omega} \left( \mathbf{F} . \mathbf{n} \right) ds +  &=&
\int_\Omega \mathbf{S} d\Omega
\ENA
where
$\mathbf{F}$ is the flux vector and
$\mathbf{n}$ is the outward pointing unit vector to the boundary $\p\Omega$. 
The flux normal flux across the face, $\mathbf{F.n}$, is given by

\begin{align}
	\mathbf{F.n}
	=
	\begin{bmatrix}
	hu_\perp  \\[.5em]
	huu_\perp + \frac{1}{2}gh^2 \cos\phi \\[.5em]
	hvu_\perp + \frac{1}{2}gh^2 \sin\phi 
	\end{bmatrix}
\end{align}
%
where $u_\perp$ is the velocity perpendicular to the face given by $u \cos\phi + v \sin\phi$ and
$\phi$ is the angle between the face normal and the x axis.


\subsection{Interface flux}
The interface flux, $\mathbf{F}_\perp$, at the face shared by cell $i$ and $j$ are evaluated using Roe's method as 

\EQ
\mathbf{F.n} \approx \mathbf{F}_\perp^{i,j} =
\frac{1}{2} \left( \mathbf{F}_\perp^{i} + \mathbf{F}_\perp^{j} - \mathbf{\hat{R}} |\mathbf{\hat{\Lambda}| \mathbf{\Delta}\hat{V}} \right)
\EN
%
where
\begin{align}
	\mathbf{R}
	=
	\begin{bmatrix}
	1 & 0 & 1  \\[.5em]
	\hat{u} - \hat{c}\cos\phi & -\sin\phi & \hat{u} + \hat{c}\cos\phi \\[.5em]
	\hat{v} - \hat{c}\sin\phi & \cos\phi & \hat{v} + \hat{c}\sin\phi
	\end{bmatrix}
\end{align}

\begin{align}
	\mathbf{\Delta\hat{V}}
	=
	\begin{bmatrix}
	\frac{1}{2} \left( \Delta h - \frac{\hat{h}\Delta u_\perp}{\hat{a}} \right) \\[.5em]
	\hat{h}u_\parallel \\[.5em]
	\frac{1}{2} \left( \Delta h + \frac{\hat{h}\Delta u_\perp}{\hat{a}} \right)
	\end{bmatrix}
\end{align}

\begin{subequations}
\EQA
  \hat{h} & = & \sqrt{h_i h_j} \\
  \hat{u} & = & \frac{ \sqrt{h_i} u_i + \sqrt{h_j} u_j}{ \sqrt{h_i} + \sqrt{h_j}} \\
  \hat{v} & = & \frac{ \sqrt{h_i} v_i + \sqrt{h_j} v_j}{ \sqrt{h_i} + \sqrt{h_j}} \\
  \hat{a} & = & \sqrt{\frac{g}{2} \left( h_i + h_j \right)}
\ENA
\end{subequations}

\begin{align}
	|\mathbf{\hat{\Lambda}}|
	=
	\begin{bmatrix}
	| \hat{u}_\perp - \hat{a} |^* & 0 & 0  \\[.5em]
	0                                     & |\hat{u}_\perp| & 0 \\[.5em]	
	0                                     &                         & | \hat{u}_\perp + \hat{a} |^* 
	\end{bmatrix}
\end{align}
The asterisks denote that the eigenvalues 
$\hat{\lambda}_1 (= \hat{u}_\perp - \hat{a} )$ and
$\hat{\lambda}_3 (= \hat{u}_\perp + \hat{a})$ 
are adjusted since Roe's method does not provide correct flux for critical flow.
\begin{subequations}
\EQA
  |\hat{\lambda}|_1 &=& \frac{\hat{\lambda}^2_1}{\Delta \lambda} + \frac{\Delta \lambda}{4} \mbox{\hspace{0.5cm}if $-\Delta \lambda/2 < \hat{\lambda}_1 < \Delta \lambda/2$} \\
  |\hat{\lambda}|_3 &=& \frac{\hat{\lambda}^2_2}{\Delta \lambda} + \frac{\Delta \lambda}{4} \mbox{\hspace{0.5cm}if $-\Delta \lambda/2 < \hat{\lambda}_3 < \Delta \lambda/2$}
\ENA
\end{subequations}
 
 \subsection{Source term}

The source term for the x-momentum equation
\EQA
\int_{d\Omega} \mathbf{S} d\Omega &=& \int_{\Omega} \left( -gh\frac{\p z}{\p x} - C_D u \sqrt{u^2 + v^2} \right) d\Omega \nonumber \\
&=& \int_{d\Omega}  -gh\frac{\p z}{\p x} d\Omega - \int_{\Omega} C_D u \sqrt{u^2 + v^2}  d\Omega 
\ENA

\subsubsection{Bed slope elevation term}

\EQA
\int_{d\Omega}  -gh\frac{\p z}{\p x} d\Omega &\approx& -gh\overline{\frac{\p z}{\p x}} \Omega \nonumber \\[0.6em]
&=& \frac{(y_2 - y_0)(z_1 - z_0) - (y_1 - y_0)(z_2 - z_0)}{(y_2 - y_0)(x_1 - x_0) - (y_1 - y_0)(x_2 - x_0)} \Omega
\ENA

\subsubsection{Roughness term}

\EQA
\int_{\Omega} C_D u \sqrt{u^2 + v^2}  d\Omega  &\approx& C_D u \sqrt{u^2 + v^2} \Omega
\ENA

\section{PETSc TS implementation}
%The equation~\ref{eq:swe} is implemented using 
The PETSc TS notation\footnote{https://petsc.org/release/docs/manual/ts/} for solving time-dependent is given as:

\EQ
\label{eqn:petsc_ts}
\mathtt{ F(t,u,{u}_t)=G(t,u)}
\EN
%
By comparing \ref{eqn:swe} and \ref{eqn:petsc_ts},
\EQA
{\mathtt{IFunction: \hspace{0.5cm} F(t,u,{u}_t)}  } & \equiv & \frac{\p\mathbf{U}}{\p t} \\
{\mathtt{RHSFunction: \hspace{1cm} G(t,u)}  } & \equiv & -\frac{\p \mathbf{E}}{\p x} - \frac{\p \mathbf{G}}{\p x} - \mathbf{S}
\ENA

\subsection{\tt{ex1.c}}

\begin{itemize}
	\item The bed is assumed flat and Manning's coefficient is assumed to be zero, so $\mathbf{S}$ is not included.
\end{itemize}


















\end{document}